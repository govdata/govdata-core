%
% API Documentation for DataEnvironment
% Module System.Web.Tabular2Html
%
% Generated by epydoc 3.0.1
% [Tue Mar 31 17:17:20 2009]
%

%%%%%%%%%%%%%%%%%%%%%%%%%%%%%%%%%%%%%%%%%%%%%%%%%%%%%%%%%%%%%%%%%%%%%%%%%%%
%%                          Module Description                           %%
%%%%%%%%%%%%%%%%%%%%%%%%%%%%%%%%%%%%%%%%%%%%%%%%%%%%%%%%%%%%%%%%%%%%%%%%%%%

    \index{System \textit{(package)}!System.Web \textit{(package)}!System.Web.Tabular2Html \textit{(module)}|(}
\section{Module System.Web.Tabular2Html}

    \label{System:Web:Tabular2Html}

%%%%%%%%%%%%%%%%%%%%%%%%%%%%%%%%%%%%%%%%%%%%%%%%%%%%%%%%%%%%%%%%%%%%%%%%%%%
%%                               Functions                               %%
%%%%%%%%%%%%%%%%%%%%%%%%%%%%%%%%%%%%%%%%%%%%%%%%%%%%%%%%%%%%%%%%%%%%%%%%%%%

  \subsection{Functions}

    \label{System:Web:Tabular2Html:Tabular2Html}
    \index{System \textit{(package)}!System.Web \textit{(package)}!System.Web.Tabular2Html \textit{(module)}!System.Web.Tabular2Html.Tabular2Html \textit{(function)}}

    \vspace{0.5ex}

\hspace{.8\funcindent}\begin{boxedminipage}{\funcwidth}

    \raggedright \textbf{Tabular2Html}(\textit{htmlFile}, \textit{ddata}={\tt None}, \textit{FileInName}={\tt None}, \textit{FirstLineInSVIsHeader}={\tt True}, \textit{Title}={\tt None})

    \vspace{-1.5ex}

    \rule{\textwidth}{0.5\fboxrule}
\setlength{\parskip}{2ex}
    Creates an html representation of a Tabular data, including .data, 
    .csv, and .tsv formats

\setlength{\parskip}{1ex}
    \end{boxedminipage}

    \label{System:Web:Tabular2Html:FixCSSName}
    \index{System \textit{(package)}!System.Web \textit{(package)}!System.Web.Tabular2Html \textit{(module)}!System.Web.Tabular2Html.FixCSSName \textit{(function)}}

    \vspace{0.5ex}

\hspace{.8\funcindent}\begin{boxedminipage}{\funcwidth}

    \raggedright \textbf{FixCSSName}(\textit{k})

\setlength{\parskip}{2ex}
\setlength{\parskip}{1ex}
    \end{boxedminipage}

    \label{System:Web:Tabular2Html:CSSColoring}
    \index{System \textit{(package)}!System.Web \textit{(package)}!System.Web.Tabular2Html \textit{(module)}!System.Web.Tabular2Html.CSSColoring \textit{(function)}}

    \vspace{0.5ex}

\hspace{.8\funcindent}\begin{boxedminipage}{\funcwidth}

    \raggedright \textbf{CSSColoring}(\textit{names}, \textit{coloring})

\setlength{\parskip}{2ex}
\setlength{\parskip}{1ex}
    \end{boxedminipage}

    \label{System:Web:Tabular2Html:ColorScheme}
    \index{System \textit{(package)}!System.Web \textit{(package)}!System.Web.Tabular2Html \textit{(module)}!System.Web.Tabular2Html.ColorScheme \textit{(function)}}

    \vspace{0.5ex}

\hspace{.8\funcindent}\begin{boxedminipage}{\funcwidth}

    \raggedright \textbf{ColorScheme}(\textit{NTree}, \textit{B}, \textit{Total})

\setlength{\parskip}{2ex}
\setlength{\parskip}{1ex}
    \end{boxedminipage}

    \label{System:Web:Tabular2Html:HeaderNotations}
    \index{System \textit{(package)}!System.Web \textit{(package)}!System.Web.Tabular2Html \textit{(module)}!System.Web.Tabular2Html.HeaderNotations \textit{(function)}}

    \vspace{0.5ex}

\hspace{.8\funcindent}\begin{boxedminipage}{\funcwidth}

    \raggedright \textbf{HeaderNotations}(\textit{names}, \textit{sdict})

\setlength{\parskip}{2ex}
\setlength{\parskip}{1ex}
    \end{boxedminipage}

    \label{System:Web:Tabular2Html:GroupByLevel}
    \index{System \textit{(package)}!System.Web \textit{(package)}!System.Web.Tabular2Html \textit{(module)}!System.Web.Tabular2Html.GroupByLevel \textit{(function)}}

    \vspace{0.5ex}

\hspace{.8\funcindent}\begin{boxedminipage}{\funcwidth}

    \raggedright \textbf{GroupByLevel}(\textit{NTree})

\setlength{\parskip}{2ex}
\setlength{\parskip}{1ex}
    \end{boxedminipage}

    \label{System:Web:Tabular2Html:MakeLine}
    \index{System \textit{(package)}!System.Web \textit{(package)}!System.Web.Tabular2Html \textit{(module)}!System.Web.Tabular2Html.MakeLine \textit{(function)}}

    \vspace{0.5ex}

\hspace{.8\funcindent}\begin{boxedminipage}{\funcwidth}

    \raggedright \textbf{MakeLine}(\textit{names}, \textit{L}, \textit{sdict})

\setlength{\parskip}{2ex}
\setlength{\parskip}{1ex}
    \end{boxedminipage}

    \label{System:Web:Tabular2Html:WriteOutCSS}
    \index{System \textit{(package)}!System.Web \textit{(package)}!System.Web.Tabular2Html \textit{(module)}!System.Web.Tabular2Html.WriteOutCSS \textit{(function)}}

    \vspace{0.5ex}

\hspace{.8\funcindent}\begin{boxedminipage}{\funcwidth}

    \raggedright \textbf{WriteOutCSS}(\textit{ColorStyles}, \textit{outpath})

\setlength{\parskip}{2ex}
\setlength{\parskip}{1ex}
    \end{boxedminipage}

    \label{System:Web:Tabular2Html:Tabular2HtmlString}
    \index{System \textit{(package)}!System.Web \textit{(package)}!System.Web.Tabular2Html \textit{(module)}!System.Web.Tabular2Html.Tabular2HtmlString \textit{(function)}}

    \vspace{0.5ex}

\hspace{.8\funcindent}\begin{boxedminipage}{\funcwidth}

    \raggedright \textbf{Tabular2HtmlString}(\textit{FileInName}, \textit{FirstLineInSVIsHeader}={\tt True})

    \vspace{-1.5ex}

    \rule{\textwidth}{0.5\fboxrule}
\setlength{\parskip}{2ex}
    Creates an html representation of a Tabular data, including .data, 
    .csv, and .tsv formats

\setlength{\parskip}{1ex}
    \end{boxedminipage}


%%%%%%%%%%%%%%%%%%%%%%%%%%%%%%%%%%%%%%%%%%%%%%%%%%%%%%%%%%%%%%%%%%%%%%%%%%%
%%                           Class Description                           %%
%%%%%%%%%%%%%%%%%%%%%%%%%%%%%%%%%%%%%%%%%%%%%%%%%%%%%%%%%%%%%%%%%%%%%%%%%%%

    \index{System \textit{(package)}!System.Web \textit{(package)}!System.Web.Tabular2Html \textit{(module)}!System.Web.Tabular2Html.NameTree \textit{(class)}|(}
\subsection{Class NameTree}

    \label{System:Web:Tabular2Html:NameTree}

%%%%%%%%%%%%%%%%%%%%%%%%%%%%%%%%%%%%%%%%%%%%%%%%%%%%%%%%%%%%%%%%%%%%%%%%%%%
%%                                Methods                                %%
%%%%%%%%%%%%%%%%%%%%%%%%%%%%%%%%%%%%%%%%%%%%%%%%%%%%%%%%%%%%%%%%%%%%%%%%%%%

  \subsubsection{Methods}

    \label{System:Web:Tabular2Html:NameTree:__init__}
    \index{System \textit{(package)}!System.Web \textit{(package)}!System.Web.Tabular2Html \textit{(module)}!System.Web.Tabular2Html.NameTree \textit{(class)}!System.Web.Tabular2Html.NameTree.\_\_init\_\_ \textit{(method)}}

    \vspace{0.5ex}

\hspace{.8\funcindent}\begin{boxedminipage}{\funcwidth}

    \raggedright \textbf{\_\_init\_\_}(\textit{self}, \textit{names}, \textit{sdict})

\setlength{\parskip}{2ex}
\setlength{\parskip}{1ex}
    \end{boxedminipage}

    \index{System \textit{(package)}!System.Web \textit{(package)}!System.Web.Tabular2Html \textit{(module)}!System.Web.Tabular2Html.NameTree \textit{(class)}|)}
    \index{System \textit{(package)}!System.Web \textit{(package)}!System.Web.Tabular2Html \textit{(module)}|)}
