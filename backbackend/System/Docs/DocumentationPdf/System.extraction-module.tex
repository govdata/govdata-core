%
% API Documentation for DataEnvironment
% Module System.extraction
%
% Generated by epydoc 3.0.1
% [Tue Mar 31 17:17:20 2009]
%

%%%%%%%%%%%%%%%%%%%%%%%%%%%%%%%%%%%%%%%%%%%%%%%%%%%%%%%%%%%%%%%%%%%%%%%%%%%
%%                          Module Description                           %%
%%%%%%%%%%%%%%%%%%%%%%%%%%%%%%%%%%%%%%%%%%%%%%%%%%%%%%%%%%%%%%%%%%%%%%%%%%%

    \index{System \textit{(package)}!System.extraction \textit{(module)}|(}
\section{Module System.extraction}

    \label{System:extraction}

%%%%%%%%%%%%%%%%%%%%%%%%%%%%%%%%%%%%%%%%%%%%%%%%%%%%%%%%%%%%%%%%%%%%%%%%%%%
%%                               Functions                               %%
%%%%%%%%%%%%%%%%%%%%%%%%%%%%%%%%%%%%%%%%%%%%%%%%%%%%%%%%%%%%%%%%%%%%%%%%%%%

  \subsection{Functions}

    \label{System:extraction:Extract}
    \index{System \textit{(package)}!System.extraction \textit{(module)}!System.extraction.Extract \textit{(function)}}

    \vspace{0.5ex}

\hspace{.8\funcindent}\begin{boxedminipage}{\funcwidth}

    \raggedright \textbf{Extract}(\textit{Seed}, \textit{ToName})

    \vspace{-1.5ex}

    \rule{\textwidth}{0.5\fboxrule}
\setlength{\parskip}{2ex}
\begin{alltt}

Given a seed path (or list of paths), extracts out all raw data in that path, 
as well as scripts necessary to produce downstream computed data from that raw data.   

The extracted object is just a single "top-level directory" containing files and 
directories copied from the file system; the directory substructure within the 
top-level directory replicates the directory structure of the original files. 

Seed = path or list of paths given as a string, or comma-separated list of path strings, 
or python list of strings.  E.g.:
                        '../Data/Dan\_Data/RandomData/' 
        or              '../Data/Dan\_Data/RandomData/,../Data/Dan\_Data/NPR\_Puzzle\_Solutions/'
        or              ['../Data/Dan\_Data/RandomData/', '../Data/Dan\_Data/NPR\_Puzzle\_Solutions/']

ToName = Name of top-level directory where extracted files will be copied.   
This function creates the path ToName if it doesn't exist, or overwrites it if it does. 
        
\end{alltt}

\setlength{\parskip}{1ex}
    \end{boxedminipage}

    \label{System:extraction:Integrate}
    \index{System \textit{(package)}!System.extraction \textit{(module)}!System.extraction.Integrate \textit{(function)}}

    \vspace{0.5ex}

\hspace{.8\funcindent}\begin{boxedminipage}{\funcwidth}

    \raggedright \textbf{Integrate}(\textit{FolderToIntegrate})

    \vspace{-1.5ex}

    \rule{\textwidth}{0.5\fboxrule}
\setlength{\parskip}{2ex}
    Suppose one has done an extraction (e.g., applied the Extract function)
    of some files in one data environment, and then taken that extracted 
    directory and put it in the Temp directory of a new, different data 
    environment.  This function is meant to integrate the extraction into 
    the new file system so that it mirros the placement of the files in the
    original data environment.

    The top-level extracted directory is assumed to be in the "Temp" 
    directory of the new data environment. This top-level directory 
    contains a replica of a subportion of the original file system from 
    which the extraction was done (see comments to Extract function). The 
    Integrate function merely copies files from the extraction directory to
    the corresponding place in the filesystem. As it does so, it asks 
    whether to continue if it determines that the path to which it would 
    copy already exists in the new data environment.

    FolderToIntegrate = path name of top-level extraction directory. 
    Required to be in "Temp".

\setlength{\parskip}{1ex}
    \end{boxedminipage}

    \label{System:extraction:Raw}
    \index{System \textit{(package)}!System.extraction \textit{(module)}!System.extraction.Raw \textit{(function)}}

    \vspace{0.5ex}

\hspace{.8\funcindent}\begin{boxedminipage}{\funcwidth}

    \raggedright \textbf{Raw}(\textit{Seed})

    \vspace{-1.5ex}

    \rule{\textwidth}{0.5\fboxrule}
\setlength{\parskip}{2ex}
\begin{alltt}

Given a seed path (or list of paths), determines a list of all raw (non-computed) 
data files in that path,
as well as scripts necessary to produce downstream computed data from that 
raw data.  

Seed = path or list of paths given as a string, or comma-separated list of path strings, 
or python list of strings.  E.g.:
                '../Data/Dan\_Data/RandomData/' 
or              '../Data/Dan\_Data/RandomData/,../Data/Dan\_Data/NPR\_Puzzle\_Solutions/'
or              ['../Data/Dan\_Data/RandomData/', '../Data/Dan\_Data/NPR\_Puzzle\_Solutions/']


Returns: [A,B] where 
--A = list of path names in the data environment; this list contains all raw 
(non-computed) data files required to produce the computed data files in Seed
paths, as well as the paths of all scripts that need to be run to produce the 
computed files.   
--B = boolean which is True when all paths listed in A actually exist in the file system. 
\end{alltt}

\setlength{\parskip}{1ex}
    \end{boxedminipage}

    \label{System:extraction:CopyOut}
    \index{System \textit{(package)}!System.extraction \textit{(module)}!System.extraction.CopyOut \textit{(function)}}

    \vspace{0.5ex}

\hspace{.8\funcindent}\begin{boxedminipage}{\funcwidth}

    \raggedright \textbf{CopyOut}(\textit{FilesToCopy}, \textit{ToName})

    \vspace{-1.5ex}

    \rule{\textwidth}{0.5\fboxrule}
\setlength{\parskip}{2ex}
    Copies out files from one location to a "top-level target directory" 
    within the Temp Directory of the Data Environment, preserving the 
    original internal directory structure.

    FilesToCopy  = List of files to copy, given as a python list of path 
    names.

    ToName = name of top-level directory to which the files in FilesToCopy 
    will be copied.   The top-level directory is given as a relative path, 
    and will be placed in the Temp directory of the data environment.

    If ToName does it exist, this function creates it.  If it does exist, 
    it will overwrite it.

    As files are copied into the ToName directory, original internal 
    director structure is replicated, e.g. if a file File.txt has path 
    ../A/File.txt relative to Temp, then (if it doesn't yet exist in 
    ToName), an empty directory called ToName/A will be created, and the 
    File.txt will compied to it.

\setlength{\parskip}{1ex}
    \end{boxedminipage}

    \label{System:extraction:CopyIn}
    \index{System \textit{(package)}!System.extraction \textit{(module)}!System.extraction.CopyIn \textit{(function)}}

    \vspace{0.5ex}

\hspace{.8\funcindent}\begin{boxedminipage}{\funcwidth}

    \raggedright \textbf{CopyIn}(\textit{ToCopy}, \textit{Target}, \textit{overwrite}={\tt False})

    \vspace{-1.5ex}

    \rule{\textwidth}{0.5\fboxrule}
\setlength{\parskip}{2ex}
    ToCopy = a path name. Target = a path name overwrite = Boolean

    This function copies the contents of ToCopy to Target, preserving 
    whatever file structure is in ToCopy (e.g. just the file if it is a 
    file, and the directories in it if it is a directory.)

    If Target does not exist, it creates it.   If it already exists, CopyIn
    overwrites Target if overwrite argument is True; if overwrite argument 
    is False, the function aborts.

\setlength{\parskip}{1ex}
    \end{boxedminipage}

    \index{System \textit{(package)}!System.extraction \textit{(module)}|)}
