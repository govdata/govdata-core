%
% API Documentation for DataEnvironment
% Module Operations.DotDataFunctions.DotDataListFromDirectory
%
% Generated by epydoc 3.0.1
% [Tue Mar 31 17:17:20 2009]
%

%%%%%%%%%%%%%%%%%%%%%%%%%%%%%%%%%%%%%%%%%%%%%%%%%%%%%%%%%%%%%%%%%%%%%%%%%%%
%%                          Module Description                           %%
%%%%%%%%%%%%%%%%%%%%%%%%%%%%%%%%%%%%%%%%%%%%%%%%%%%%%%%%%%%%%%%%%%%%%%%%%%%

    \index{Operations \textit{(package)}!Operations.DotDataFunctions \textit{(package)}!Operations.DotDataFunctions.DotDataListFromDirectory \textit{(module)}|(}
\section{Module Operations.DotDataFunctions.DotDataListFromDirectory}

    \label{Operations:DotDataFunctions:DotDataListFromDirectory}

%%%%%%%%%%%%%%%%%%%%%%%%%%%%%%%%%%%%%%%%%%%%%%%%%%%%%%%%%%%%%%%%%%%%%%%%%%%
%%                               Functions                               %%
%%%%%%%%%%%%%%%%%%%%%%%%%%%%%%%%%%%%%%%%%%%%%%%%%%%%%%%%%%%%%%%%%%%%%%%%%%%

  \subsection{Functions}

    \label{Operations:DotDataFunctions:DotDataListFromDirectory:DotDataListFromDirectory}
    \index{Operations \textit{(package)}!Operations.DotDataFunctions \textit{(package)}!Operations.DotDataFunctions.DotDataListFromDirectory \textit{(module)}!Operations.DotDataFunctions.DotDataListFromDirectory.DotDataListFromDirectory \textit{(function)}}

    \vspace{0.5ex}

\hspace{.8\funcindent}\begin{boxedminipage}{\funcwidth}

    \raggedright \textbf{DotDataListFromDirectory}(\textit{path}, \textit{data\_list}={\tt None}, \textit{attribute\_names}={\tt None}, \textit{rootpath}={\tt ''}, \textit{rootheader}={\tt None}, \textit{coloring}={\tt None}, \textit{ToLoad}={\tt None}, \textit{Nrecords}={\tt None})

    \vspace{-1.5ex}

    \rule{\textwidth}{0.5\fboxrule}
\setlength{\parskip}{2ex}
\begin{alltt}
Load DotData from a .data/ directory.  
Each column is stored in a separate file 
        (for column named myColumn the file is myColumn.datatype.csv
        where datatype is the data type of the column data ( int, int64, float, str, ... ), 
        and the ordered list of columns is in a separate header file.)

See also: SaveDotData()
\end{alltt}

\setlength{\parskip}{1ex}
    \end{boxedminipage}

    \index{Operations \textit{(package)}!Operations.DotDataFunctions \textit{(package)}!Operations.DotDataFunctions.DotDataListFromDirectory \textit{(module)}|)}
