%
% API Documentation for DataEnvironment
% Module System.system_io_override
%
% Generated by epydoc 3.0.1
% [Tue Mar 31 17:17:20 2009]
%

%%%%%%%%%%%%%%%%%%%%%%%%%%%%%%%%%%%%%%%%%%%%%%%%%%%%%%%%%%%%%%%%%%%%%%%%%%%
%%                          Module Description                           %%
%%%%%%%%%%%%%%%%%%%%%%%%%%%%%%%%%%%%%%%%%%%%%%%%%%%%%%%%%%%%%%%%%%%%%%%%%%%

    \index{System \textit{(package)}!System.system\_io\_override \textit{(module)}|(}
\section{Module System.system\_io\_override}

    \label{System:system_io_override}
\begin{alltt}

Intercept system i/o functions and analyze stack that made i/o calls. 

One of the main purposes of the Data Environment to understand 
and manage data dependencies.  At a basic level, these data 
dependencies involve reading and writing files to disk.  To this end, 
it is useful to be able to determine exactly which files are being read 
and written, realtime -- by during processes of individual users doing 
exploratory data analysis as well as by automatic system updates.  
One wants not only to know which i/o streams are being started, but
also which high-level processes and functions are doing the reading and writing.    

This module provides routines for:
        1) intercepting i/o streams at a low level
        2) analyzing the call stack above those streams
        3) integrating the information from 1 and 2 to:
                -- store data about which files were read/written, and 
                therefore enable run-time determination of \_undeclared\_ 
                data dependency links
        and 
                -- impose controls on which files can be written by which 
                processes, thereby enabling the enforcement of \_declared\_ 
                data dependency links. 
        
The idea behind the imposition of controls is not to give the Data Enviroment 
an account security interface whereby Data Environment users can be 
flexibly prevented by Data Environment administrators from doing things
with files that don't belong to them. (Such things are probably better 
administered at the operating system level by real user accounts.)  
Instead, the idea is to allow the users to impose controls on themselves 
so that they, and others using their data, can be assured that the 
LinkLists that the Data Environment builds from the users' declarations
are an accurate reflection of the way the data was actually produced 
-- and have errors raised when dependency violations occur.  This allows 
users to easily see where they've forgotten to declare dependencies properly, 
and trivially know how to remedy the issue. 

The majority of the routines in this file are the system i/o intercepts themselves.  
The idea in each case is to implement for a standard i/o function, the
following three-step "redefinition" procedure: 
        1) take the standard \_\_builtin\_\_ I/O functions and save it to a "hold name" 
                (e.g. by prefixing "old\_" to the name)
        2) redefine a new i/o function with PRECISELY the same interface as 
                the old one, and whichn calls the function at the "holding name" after
                first having passed the inputs and outputs through an I/O interception 
                and check that uses information from the GetDependsCreates() call 
                stack analyzer function. 
        3) redefine the system's standard I/O name as the new i/o function.
        
Step 2) can either implement:
        a) a "control" -- meaning that an i/o function that is called without being 
                covered by a declared dependency will raise a "BadCheckError" that
                prevents the I/O operation from occuring and prints information about the 
                violation, 
or 
        b) a "logging" operation -- meaning that data about the i/o request, together 
                with the stack inforamtion about where it was made and whether it was
                consistent with declarations -- is stored in some standard format and place.
-- and of course, it can do both.  

The slightly complex 3-step redefinition procedure is used to prevent 
problematic infinite recursions.  It also must be handled with care because,
once these intercepts are activated, ALL calls to these system functions 
will be passed through the intercepts.   This means that:
        -- protections can't be set too widely,
        -- interfaces of the functions must be preversed EXACTLY,
so that programs indirected called by the user as dependencies to 
other functions, and that require resources "outside" the system, are 
not bizarrely restricted from performing their work. (This is especially 
the case for automatic logging and other interactive features of enhanced 
python shells like iPython, which I tend to want to use.) 
\end{alltt}


%%%%%%%%%%%%%%%%%%%%%%%%%%%%%%%%%%%%%%%%%%%%%%%%%%%%%%%%%%%%%%%%%%%%%%%%%%%
%%                               Functions                               %%
%%%%%%%%%%%%%%%%%%%%%%%%%%%%%%%%%%%%%%%%%%%%%%%%%%%%%%%%%%%%%%%%%%%%%%%%%%%

  \subsection{Functions}

    \label{System:system_io_override:IsntProtected}
    \index{System \textit{(package)}!System.system\_io\_override \textit{(module)}!System.system\_io\_override.IsntProtected \textit{(function)}}

    \vspace{0.5ex}

\hspace{.8\funcindent}\begin{boxedminipage}{\funcwidth}

    \raggedright \textbf{IsntProtected}(\textit{r})

    \vspace{-1.5ex}

    \rule{\textwidth}{0.5\fboxrule}
\setlength{\parskip}{2ex}
    If the environment variable "PROTECTION" is set to "ON" (see comments 
    in i/o GetDependsCreates), this function looks to see whether an 
    environment variable called "ProtectedPathsListFile' has been set. This
    is a path to a user configuration file which would list, in lines, the 
    list of directories on which the user wishes i/o protection to be 
    enforced.   If no such path variable is set, the system looks uses 
    DataEnvironmentDirectory as the default, e.g everything in the 
    DataEnvironment would be protected if the PROTECTION variable is set at
    'ON'.

\setlength{\parskip}{1ex}
    \end{boxedminipage}

    \label{System:system_io_override:GetDependsCreates}
    \index{System \textit{(package)}!System.system\_io\_override \textit{(module)}!System.system\_io\_override.GetDependsCreates \textit{(function)}}

    \vspace{0.5ex}

\hspace{.8\funcindent}\begin{boxedminipage}{\funcwidth}

    \raggedright \textbf{GetDependsCreates}()

    \vspace{-1.5ex}

    \rule{\textwidth}{0.5\fboxrule}
\setlength{\parskip}{2ex}
\begin{alltt}

This function performs call stack analysis looking to capture and 
aggregate the settings of the two protected key-word variables 
"depends\_on' and "creates" that are present up the stack. 

RETURNS:
--[dlist,clist] where:
If the "PROTECTION" variable is set to 'ON':
        dlist is the aggregate tuple of depends\_on values seen 
        up the stack tree, and clist is the aggregate tuple of 
        creates values. 
        
        If 'SystemMode' environment variable is set at 'Exploratory', 
        it adds '../' to the dlist, (which will be interpreted to allow
        unlimited read access) -- the idea being that at the user prompt, 
        mostly the user will want to be able to at least read whatever 
        files he/she wants, even he/she cannot then write back to those
        paths.   
        
else:
        clist and dlist are set to '../', which will be interpreted by system 
        i/o intercepting functions that call GetDependsCreates to mean 
        that unlimited read/write access is granted. 
        
'../Temp' is appended to clist and dlist in all cases, to allow the user 
to read and write whatever is desired in working Temp directory. 
        
Probably a more flexible and less hard-coded way of allowing the
user to set these read-write protections would be useful in the future.
\end{alltt}

\setlength{\parskip}{1ex}
    \end{boxedminipage}

    \label{System:system_io_override:system_open}
    \index{System \textit{(package)}!System.system\_io\_override \textit{(module)}!System.system\_io\_override.system\_open \textit{(function)}}

    \vspace{0.5ex}

\hspace{.8\funcindent}\begin{boxedminipage}{\funcwidth}

    \raggedright \textbf{system\_open}(\textit{ToOpen}, \textit{Mode}={\tt 'r'}, \textit{bufsize}={\tt 1})

\setlength{\parskip}{2ex}
\setlength{\parskip}{1ex}
    \end{boxedminipage}

    \label{System:system_io_override:system_copy}
    \index{System \textit{(package)}!System.system\_io\_override \textit{(module)}!System.system\_io\_override.system\_copy \textit{(function)}}

    \vspace{0.5ex}

\hspace{.8\funcindent}\begin{boxedminipage}{\funcwidth}

    \raggedright \textbf{system\_copy}(\textit{tocopy}, \textit{destination})

\setlength{\parskip}{2ex}
\setlength{\parskip}{1ex}
    \end{boxedminipage}

    \label{System:system_io_override:system_copy2}
    \index{System \textit{(package)}!System.system\_io\_override \textit{(module)}!System.system\_io\_override.system\_copy2 \textit{(function)}}

    \vspace{0.5ex}

\hspace{.8\funcindent}\begin{boxedminipage}{\funcwidth}

    \raggedright \textbf{system\_copy2}(\textit{tocopy}, \textit{destination})

\setlength{\parskip}{2ex}
\setlength{\parskip}{1ex}
    \end{boxedminipage}

    \label{System:system_io_override:system_copytree}
    \index{System \textit{(package)}!System.system\_io\_override \textit{(module)}!System.system\_io\_override.system\_copytree \textit{(function)}}

    \vspace{0.5ex}

\hspace{.8\funcindent}\begin{boxedminipage}{\funcwidth}

    \raggedright \textbf{system\_copytree}(\textit{tocopy}, \textit{destination}, \textit{symlinks}={\tt False})

\setlength{\parskip}{2ex}
\setlength{\parskip}{1ex}
    \end{boxedminipage}

    \label{System:system_io_override:system_mkdir}
    \index{System \textit{(package)}!System.system\_io\_override \textit{(module)}!System.system\_io\_override.system\_mkdir \textit{(function)}}

    \vspace{0.5ex}

\hspace{.8\funcindent}\begin{boxedminipage}{\funcwidth}

    \raggedright \textbf{system\_mkdir}(\textit{DirName}, \textit{mode}={\tt 0777})

\setlength{\parskip}{2ex}
\setlength{\parskip}{1ex}
    \end{boxedminipage}

    \label{System:system_io_override:system_makedirs}
    \index{System \textit{(package)}!System.system\_io\_override \textit{(module)}!System.system\_io\_override.system\_makedirs \textit{(function)}}

    \vspace{0.5ex}

\hspace{.8\funcindent}\begin{boxedminipage}{\funcwidth}

    \raggedright \textbf{system\_makedirs}(\textit{DirName}, \textit{mode}={\tt 0777})

\setlength{\parskip}{2ex}
\setlength{\parskip}{1ex}
    \end{boxedminipage}

    \label{System:system_io_override:system_rename}
    \index{System \textit{(package)}!System.system\_io\_override \textit{(module)}!System.system\_io\_override.system\_rename \textit{(function)}}

    \vspace{0.5ex}

\hspace{.8\funcindent}\begin{boxedminipage}{\funcwidth}

    \raggedright \textbf{system\_rename}(\textit{old}, \textit{new})

\setlength{\parskip}{2ex}
\setlength{\parskip}{1ex}
    \end{boxedminipage}

    \label{System:system_io_override:system_remove}
    \index{System \textit{(package)}!System.system\_io\_override \textit{(module)}!System.system\_io\_override.system\_remove \textit{(function)}}

    \vspace{0.5ex}

\hspace{.8\funcindent}\begin{boxedminipage}{\funcwidth}

    \raggedright \textbf{system\_remove}(\textit{ToDelete})

\setlength{\parskip}{2ex}
\setlength{\parskip}{1ex}
    \end{boxedminipage}

    \label{System:system_io_override:system_rmtree}
    \index{System \textit{(package)}!System.system\_io\_override \textit{(module)}!System.system\_io\_override.system\_rmtree \textit{(function)}}

    \vspace{0.5ex}

\hspace{.8\funcindent}\begin{boxedminipage}{\funcwidth}

    \raggedright \textbf{system\_rmtree}(\textit{ToDelete}, \textit{ignore\_errors}={\tt False}, \textit{onerror}={\tt None})

\setlength{\parskip}{2ex}
\setlength{\parskip}{1ex}
    \end{boxedminipage}

    \label{System:system_io_override:system_exists}
    \index{System \textit{(package)}!System.system\_io\_override \textit{(module)}!System.system\_io\_override.system\_exists \textit{(function)}}

    \vspace{0.5ex}

\hspace{.8\funcindent}\begin{boxedminipage}{\funcwidth}

    \raggedright \textbf{system\_exists}(\textit{ToCheck})

\setlength{\parskip}{2ex}
\setlength{\parskip}{1ex}
    \end{boxedminipage}

    \label{System:system_io_override:system_listdir}
    \index{System \textit{(package)}!System.system\_io\_override \textit{(module)}!System.system\_io\_override.system\_listdir \textit{(function)}}

    \vspace{0.5ex}

\hspace{.8\funcindent}\begin{boxedminipage}{\funcwidth}

    \raggedright \textbf{system\_listdir}(\textit{ToList})

\setlength{\parskip}{2ex}
\setlength{\parskip}{1ex}
    \end{boxedminipage}

    \label{System:system_io_override:system_isfile}
    \index{System \textit{(package)}!System.system\_io\_override \textit{(module)}!System.system\_io\_override.system\_isfile \textit{(function)}}

    \vspace{0.5ex}

\hspace{.8\funcindent}\begin{boxedminipage}{\funcwidth}

    \raggedright \textbf{system\_isfile}(\textit{ToCheck})

\setlength{\parskip}{2ex}
\setlength{\parskip}{1ex}
    \end{boxedminipage}

    \label{System:system_io_override:system_isdir}
    \index{System \textit{(package)}!System.system\_io\_override \textit{(module)}!System.system\_io\_override.system\_isdir \textit{(function)}}

    \vspace{0.5ex}

\hspace{.8\funcindent}\begin{boxedminipage}{\funcwidth}

    \raggedright \textbf{system\_isdir}(\textit{ToCheck})

\setlength{\parskip}{2ex}
\setlength{\parskip}{1ex}
    \end{boxedminipage}

    \label{System:system_io_override:system_getmtime}
    \index{System \textit{(package)}!System.system\_io\_override \textit{(module)}!System.system\_io\_override.system\_getmtime \textit{(function)}}

    \vspace{0.5ex}

\hspace{.8\funcindent}\begin{boxedminipage}{\funcwidth}

    \raggedright \textbf{system\_getmtime}(\textit{ToAssay})

\setlength{\parskip}{2ex}
\setlength{\parskip}{1ex}
    \end{boxedminipage}

    \label{System:system_io_override:system_getatime}
    \index{System \textit{(package)}!System.system\_io\_override \textit{(module)}!System.system\_io\_override.system\_getatime \textit{(function)}}

    \vspace{0.5ex}

\hspace{.8\funcindent}\begin{boxedminipage}{\funcwidth}

    \raggedright \textbf{system\_getatime}(\textit{ToAssay})

\setlength{\parskip}{2ex}
\setlength{\parskip}{1ex}
    \end{boxedminipage}


%%%%%%%%%%%%%%%%%%%%%%%%%%%%%%%%%%%%%%%%%%%%%%%%%%%%%%%%%%%%%%%%%%%%%%%%%%%
%%                               Variables                               %%
%%%%%%%%%%%%%%%%%%%%%%%%%%%%%%%%%%%%%%%%%%%%%%%%%%%%%%%%%%%%%%%%%%%%%%%%%%%

  \subsection{Variables}

    \vspace{-1cm}
\hspace{\varindent}\begin{longtable}{|p{\varnamewidth}|p{\vardescrwidth}|l}
\cline{1-2}
\cline{1-2} \centering \textbf{Name} & \centering \textbf{Description}& \\
\cline{1-2}
\endhead\cline{1-2}\multicolumn{3}{r}{\small\textit{continued on next page}}\\\endfoot\cline{1-2}
\endlastfoot\raggedright o\-l\-d\-\_\-o\-p\-e\-n\- & \raggedright \textbf{Value:} 
{\tt \_\_builtin\_\_.open}&\\
\cline{1-2}
\raggedright o\-l\-d\-\_\-c\-o\-p\-y\- & \raggedright \textbf{Value:} 
{\tt shutil.copy}&\\
\cline{1-2}
\raggedright o\-l\-d\-\_\-c\-o\-p\-y\-2\- & \raggedright \textbf{Value:} 
{\tt shutil.copy2}&\\
\cline{1-2}
\raggedright o\-l\-d\-\_\-c\-o\-p\-y\-t\-r\-e\-e\- & \raggedright \textbf{Value:} 
{\tt shutil.copytree}&\\
\cline{1-2}
\raggedright o\-l\-d\-\_\-m\-k\-d\-i\-r\- & \raggedright \textbf{Value:} 
{\tt os.mkdir}&\\
\cline{1-2}
\raggedright o\-l\-d\-\_\-m\-a\-k\-e\-d\-i\-r\-s\- & \raggedright \textbf{Value:} 
{\tt os.makedirs}&\\
\cline{1-2}
\raggedright o\-l\-d\-\_\-r\-e\-n\-a\-m\-e\- & \raggedright \textbf{Value:} 
{\tt os.rename}&\\
\cline{1-2}
\raggedright o\-l\-d\-\_\-r\-e\-m\-o\-v\-e\- & \raggedright \textbf{Value:} 
{\tt os.remove}&\\
\cline{1-2}
\raggedright o\-l\-d\-\_\-r\-m\-t\-r\-e\-e\- & \raggedright \textbf{Value:} 
{\tt shutil.rmtree}&\\
\cline{1-2}
\raggedright o\-l\-d\-\_\-e\-x\-i\-s\-t\-s\- & \raggedright \textbf{Value:} 
{\tt os.path.exists}&\\
\cline{1-2}
\raggedright o\-l\-d\-\_\-l\-i\-s\-t\-d\-i\-r\- & \raggedright \textbf{Value:} 
{\tt os.listdir}&\\
\cline{1-2}
\raggedright o\-l\-d\-\_\-i\-s\-f\-i\-l\-e\- & \raggedright \textbf{Value:} 
{\tt os.path.isfile}&\\
\cline{1-2}
\raggedright o\-l\-d\-\_\-i\-s\-d\-i\-r\- & \raggedright \textbf{Value:} 
{\tt os.path.isdir}&\\
\cline{1-2}
\raggedright o\-l\-d\-\_\-m\-t\-i\-m\-e\- & \raggedright \textbf{Value:} 
{\tt os.path.getmtime}&\\
\cline{1-2}
\raggedright o\-l\-d\-\_\-a\-t\-i\-m\-e\- & \raggedright \textbf{Value:} 
{\tt os.path.getatime}&\\
\cline{1-2}
\end{longtable}

    \index{System \textit{(package)}!System.system\_io\_override \textit{(module)}|)}
